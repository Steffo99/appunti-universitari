\documentclass{article}
\usepackage[utf8]{inputenc}
\usepackage{mathtools}
\usepackage{amssymb}
\usepackage{centernot}
% New symbols
\let\oldsqrt\sqrt
\def\sqrt{\mathpalette\DHLhksqrt}
\def\DHLhksqrt#1#2{%
\setbox0=\hbox{$#1\oldsqrt{#2\,}$}\dimen0=\ht0
\advance\dimen0-0.2\ht0
\setbox2=\hbox{\vrule height\ht0 depth -\dimen0}
{\box0\lower0.4pt\box2}}
% End new symbols

\begin{document}
\section{Definizione}

\[f : A \subseteq dom(f) \to \mathbb{R}, continua\]
\[A = [a, b]\]

f è derivabile in \(x_0\) se \textbf{esiste ed è finito} il limite del rapporto incrementale:
\[f'(x) = \lim_{h \to 0}\frac{f(x+h) - f(x)}{h}\]

f è il \textbf{coefficiente angolare} della retta tangente a \(f(x_0)\).

\subsection{Equazione retta tangente al grafico di f in \(x_0, f(x_0)\)}
\[y = f(x_0) + f'(x_0) * (x - x_0)\]

\section{Derivate particolari}
\(f = costante\); \(f' = 0\)\\
\(f = x\); \(f' = 1\)\\
\(f = x^2\); \(f' = 2x\)\\
\(f = x^n\); \(f' = nx^{n-1}\)\\

\subsubsection{Dimostrazione di \(x^n\)}
[todo]
\[\lim_{h \to 0} \frac{(x+h)^\alpha - x^\alpha}{h}\]
\[\lim_{h \to 0} x^\alpha * (\frac{\frac{(x+h)}{x}^\alpha - 1}{h})\]
\[\lim_{h \to 0} x^\alpha * (\frac{e^{\log(\frac{(x+h)}{x}^\alpha)} - 1}{h})\]
\[\lim_{h \to 0} x^\alpha * (\frac{e^{\alpha \log(\frac{(x+h)}{x})} - 1}{h})\]
\[\lim_{h \to 0} x^\alpha * (\frac{e^\frac{\alpha h}{x}) - 1}{h})\]

\subsubsection{Non derivate il valore assoluto}
Campagna pubblicitaria: chi deriva il valore assoluto muore (accademicamente).
\(|x|\) non è derivabile in \(x = 0\).
\[\lim_{h \to 0} \frac{|h|}{h} = \nexists\]
\[\lim_{h \to 0^+} \frac{|h|}{h} = 1\]
\[\lim_{h \to 0^-} \frac{|h|}{h} = -1\]

\section{Derivate sinistra e destra}
Derivata destra:
\[f_+'(x) = \lim_{h \to 0^+}\frac{f(x+h) - f(x)}{h}\]
Derivata sinistra:
\[f_-'(x) = \lim_{h \to 0^-}\frac{f(x+h) - f(x)}{h}\]

[todo: migliorare un po']
\begin{itemize}
\item Se sono uguali e finite, esiste la derivata in quel punto;\\
\item se sono diverse e almeno una delle due finita, si ha un \textbf{punto angoloso};\\
\item se sono diverse e infinite, la tangente esiste ed è completamente verticale;\\
\item se sono uguali e infinite, si forma una cuspide.
\end{itemize}

\section{Teorema di continuità}
Se \(f\) è derivabile in \(x_0\), allora \(f\) è continua.

\paragraph{Tesi}
\[\lim_{x \to x_0} f(x) = f(x_0)\]

\paragraph{Dimostrazione}
\[\lim_{x \to x_0} f(x) - f(x_0) = 0\]
\[\lim_{x \to x_0} (f(x) - f(x_0)) * \frac{x - x_0}{x - x_0}\]
\[\lim_{x \to x_0} \frac{f(x) - f(x_0)}{x - x_0} * (x - x_0)\]
\[f'(x_0) * 0 = 0\]

\subsection{Conseguenze}
\(f\) derivabile in \(x_0\) \(\implies\) \(f\) continua in \(x_0\)\\
\(f\) non continua \(\implies\) \(f\) non derivabile\\
\(f\) continua \(\centernot\implies\) \(f\) derivabile\\
\(f\) non derivabile \(\centernot\implies\) \(f\) non continua\\

\subsubsection{Esempio}
\[f(x) = 
\begin{cases}
    1 \qquad x > 0\\
    0 \qquad x \leq 0
\end{cases}\]

Non continua in \(x = 0\), quindi non derivabile in quel punto.\\
In tutti gli altri casi, \(f'(x) = 0\).

\section{Regole di calcolo}
\[(f + g)' = f' + g'\]
\[(kf)' = kf'\]
\[(f * g)' = (f' * g) + (f * g')\]
\[(\frac{f}{g})' = \frac{(f' * g) - (f * g')}{g^2}\]

\subsection{Regola della catena}
Se \(f\) è derivabile in \(x_0\) e g è derivabile in \(f(x_0)\) e \(x_0\) è punto di accumulazione per \(dom(g \circ f)\), allora \(g \circ f\) è derivabile in \(x_0\) e vale:
\[(g \circ f)'(x_0) = g'(f(x_0)) * f'(x_0)\]

\subsubsection{Esempio}
\[f(x) = \sin^2(4 \sqrt{x} + 2\]
\[f'(x) = 2 \sin (4 \sqrt{x} + 2) * \cos (4 \sqrt{x} + 2) * (4 * \frac{1}{2 \sqrt{x}})\]

\subsubsection{Esempio}
\[f(x) = \arctan \frac{2x}{\sqrt{x^3+1}}\]
\[f'(x) = \cfrac{1}{1 + (\cfrac{2x}{\sqrt{x^3 + 1}}} * \frac{2 \sqrt{x^3 + 1} - 2x}{x^3+1} * \frac{3 x^2}{2} * \frac{1}{\sqrt{x^3+1}}\]

\subsection{Derivata della funzione inversa}
\(f : (a, b) \to \mathbb{R}\) continua e strettamente monotona \(\implies f\) invertibile\\
\(f^{-1}\) funzione di \(f\)\\
\(f\) derivabile in \(x_0 \implies f^{-1}\) derivabile in \(f(x_0) = y_0\)\\\\
\((f^{-1})'(y_0) = \frac{1}{f'(x_0)}\)\\

\subsubsection{Esempio}
\[f(x) = x + e^x\]
\[\exists f^{-1}\]
Determinare l'equazione della tangente al grafico di \(f^{-1}\) in (1, 0).\\

\[y_0 = f(x_0) = 0 + e^0 = 1\]
\[x_0 = f^{-1}(y_0) = 0\]

\[f^{-1}'(y_0) = \frac{1}{1 + e^x}\]
\[y - f^{-1}(y_0) = (f^{-1})'(y_0) * (x - y_0)\]
\[y - 0 = \frac{1}{1 + e^{1}} * (x - 1)\]
\[y = \frac{1}{1 + e} * (x - 1)\]

\section{O piccolo}
Date due funzioni \(f\) e \(g\) definite in un intorno di \(x_0\), diciamo che \(f(x) = o(g(x))\), f è \textbf{o piccolo} di \(g\) per \(x \to x_0\) se \(\lim_{x \to x_0} \frac{f(x)}{g(x)} = 0\).

\subsubsection{Esempio}
\[x^2 = o(x) \qquad x \to 0\]
Sì, perchè \(\lim_{x \to 0} \frac{x^2}{x} = 0\).

\subsubsection{Esempio}
\[sin x = o(x) \qquad x \to 0\]
No, perchè \(\lim_{x \to 0} \frac{sin x}{x} = 1\).

\subsection{Proposizione}
\[f(x) \sim g(x) \Rleftarrow f(x) = g(x) + o(g(x)) \Rleftarrow \lim_{x \to x_0} \frac{f(x)}{g(x)} = 1 \Rleftarrow \lim_{x \to x_0} - 1 = 0 \Rleftarrow \lim_{x \to x_0} \frac{f(x) - g(x)}{g(x)} = 0\]

\end{document}
