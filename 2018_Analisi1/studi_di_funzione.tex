\documentclass{article}
\usepackage[utf8]{inputenc}
\usepackage{mathtools}
\usepackage{amssymb}
\usepackage{centernot}
% New symbols
\let\oldsqrt\sqrt
\def\sqrt{\mathpalette\DHLhksqrt}
\def\DHLhksqrt#1#2{%
\setbox0=\hbox{$#1\oldsqrt{#2\,}$}\dimen0=\ht0
\advance\dimen0-0.2\ht0
\setbox2=\hbox{\vrule height\ht0 depth -\dimen0}
{\box0\lower0.4pt\box2}}
\newcommand{\iu}{\mathrm{i}\mkern1mu}
% End new symbols

\begin{document}

\section{Studi di funzione}

\subsection{Studio di funzione classico}

\[f(x) = 2 arctan(x) - x\]

\subsubsection{Funzione \(f\)}

\paragraph{Dominio}
Il dominio in un punto è il più grande insieme possibile su cui è valida la funzione \(f\).\\\\
In questo caso, il dominio è \(\mathbb{R}\)

\paragraph{Simmetrie}
Verifichiamo se la funzione ha simmetrie: è pari? È dispari?\\\\
\(arctan(x)\) è dispari, e \(x\) è anch'esso dispari, quindi andiamo a verificare.\\
\[f(-x) = 2 arctan(-x) + x = -2 arctan(x) + x = -f(x)\]
E' dunque dispari.

\paragraph{Positività}
Troviamo dove la funzione è positiva o negativa.\\
Spesso richiede calcoli molto complessi, quindi potrebbe non valer la pena perderci tempo.\\\\
Ad esempio, in questo caso.

\paragraph{Periodicità}
Controlliamo se e dove la funzione è periodica.\\
Come per la positività, potrebbe richiedere calcoli complessi, quindi non è particolarmente importante.\\\\
Come qui.

\paragraph{Intersezioni con gli assi}
Troviamo dove la funzione \(f\) interseca gli assi \(x\) e \(y\).\\
Vedi sopra; non è fondamentale...\\\\
E indovina un po'? Anche qui lo saltiamo.

\paragraph{Asintoti verticali e orizzontali}
Vediamo se la funzione ha degli asintoti.\\
Troviamo tutti i limiti rilevanti di \(f\).\\
A \(+\infty\) e a \(-\infty\), in punti di non derivabilità, etc...\\\\
\[\lim_{x \to +\infty} (2 arctan(x) - x) = -\infty\]
Essendo una funzione dispari, allora...
\[\lim_{x \to -\infty} f(x) = +\infty\]

\paragraph{Asintoto obliquo}
Controlliamo se esiste un asintoto obliquo.\\
Non è fondamentale, ma potrebbe essere interessante da calcolare.\\
E' presente solo se \(\lim_{x \to \infty} = \pm\infty\).\\
Se lo fa, possiamo calcolarlo.\\\\
\[m = \lim_{x \to +\infty} \frac{2 arctan(x) - x}{x} = -1\]
\[q = \lim_{x \to +\infty} (2 arctan(x) - x) + x = \pi\]
Dunque, l'asintoto obliquo è la retta \(y = -x + \pi\).

\subsubsection{Derivata prima \(f'\)}
\[f'(x) = \frac{2}{1 + x^2} - 1\]

\paragraph{Crescenza}
Troviamo dove la funzione è crescente o decrescente.\\
\[\frac{2 - 1 - x^2}{1 + x^2} \geq 0\]
\[\frac{1 - x^2}{1 + x^2} \geq 0\]
\[x^2 \leq 1\]
\[-1 \leq x \leq 1\]

\paragraph{Punti di estremo}
Troviamo i punti di massimo e i punti di minimo, e se possibile il loro valore.\\\\
Nel nostro caso, \(x = -1\) è un punto di minimo locale e \(x = 1\) è un punto di massimo locale.\\
Vediamo quanto valgono:
\[f(1) = 2 arctan(1) - 1 = 2 \frac{\pi}{4} - 1 = \frac{\pi - 2}{2} \approx 0.6\]
\[f(-1) = 2 arctan(-1) + 1 = - 2 \frac{\pi}{4} + 1 = \frac{- \pi + 2}{2} \approx -0.6\]

\subsubsection{Derivata seconda \(f''\)}
Potrebbe non essere richiesta, se si creerebbe un calcolo complicato.
\[f''(x) = -\frac{4x}{(1 + x^2)^2}\]

\paragraph{Concavità}
Troviamo dove la funzione è concava e dove è convessa.\\
\[-\frac{4x}{(1 + x^2)^2} \geq 0\]
\[x \geq 0\]

\paragraph{Punti di flesso}
Troviamo i punti di flesso:\\\\
Nel nostro caso, l'unico è \(x = 0\).

\subsection{Esercizio}
Fai un grafico qualitativo di \(log | 4 - x | + \frac{2}{|x - 4|}\).

\paragraph{Simmetrie}
E' simmetrica per l'asse \(x = 4\), ma il punto nell'asse stesso è fuori dal dominio.
Possiamo però traslare il tutto ponendo \(x - 4 = t\)...
\[f(t) = \log(|t|) + \frac{2}{|t|}\].
Ora la funzione \(f(t)\) è pari.

\paragraph{Dominio}
\[{t \in \mathbb{R} : t \neq 0}\]

\paragraph{Positività}
\begin{quote}
    E' un casino!
\end{quote}

\paragraph{Limiti}
[todo]

\subsubsection{Derivata prima}
\begin{quote}
    Il valore assoluto è una specie protetta; gli informatici non hanno la licenza di derivarlo.
\end{quote}
Dividiamo la funzione in casi.
\[\tilde{f}(t) = \begin{cases}
    \log t + \frac{2}{t} \qquad t > 0\\
    \log (-t) - \frac{2}{t} \qquad t < 0
\end{cases}\]
Deriviamo i due rami separatamente:
\[\tilde{f}'(t) = \begin{cases}
    \frac{1}{t} - \frac{2}{t^2} \qquad t > 0\\
    [todo] \qquad t < 0
\end{cases}\]

\subsection{Studio di funzione qualitativo in un punto}
Esiste, ma non l'abbiamo fatto.

\end{document}