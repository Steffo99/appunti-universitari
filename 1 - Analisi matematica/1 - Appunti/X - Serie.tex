\documentclass{article}
\usepackage[utf8]{inputenc}
\usepackage{mathtools}
\usepackage{amssymb}
\usepackage{centernot}
% New symbols
\let\oldsqrt\sqrt
\def\sqrt{\mathpalette\DHLhksqrt}
\def\DHLhksqrt#1#2{%
\setbox0=\hbox{$#1\oldsqrt{#2\,}$}\dimen0=\ht0
\advance\dimen0-0.2\ht0
\setbox2=\hbox{\vrule height\ht0 depth -\dimen0}
{\box0\lower0.4pt\box2}}
% End new symbols

\begin{document}

\section{Le Serie}

\[\sum^{\infty}_{n=0}a_n\]

Se \(\sum^{\infty}_{n=0}a_n = L\), la serie è \textbf{convergente}; se \(\sum^{\infty}_{n=0}a_n = \infty\), la serie è \textbf{divergente}.

\subsection{Condizione necessaria}
\[\sum^{\infty}_{n=0}a_n < +\infty \quad \implies \quad a_n \to 0\]
\[a_n \not\to 0 \quad \implies \quad \sum^{\infty}_{n=0}a_n non convergente\]

\subsection{Serie a termini non negativi definitivamente}
\[\sum^{\infty}_{n=0}a_n \qquad a_n \geq 0\]
Se la successione delle somme parziali è \textit{definitivamente} monotona, allora \textbf{ha limite}, e quindi \textbf{esiste}, convergendo o divergendo.\\
Possiamo applicare dei particolari criteri per capirlo.

\subsection{Criteri}

\subsubsection{Criterio del confronto}
Siano \(\{a_n\}\) e \(\{b_n\}\) due successioni a termini reali \textit{non negativi}, tali che \textit{definitivamente} \(a_n \leq b_n\).\\
Allora...
\[\sum^{\infty}_{n=0} b_n < +\infty \implies \sum a_n < +\infty\]
\[\sum^{\infty}_{n=0} a_n = +\infty \implies \sum b_n = +\infty\]
Si usa principalmente quando la serie converge ma non è dimostrabile convenzionalmente.

\subsubsection{Criterio del confronto asintotico}
Siano \(\{a_n\}\) e \(\{b_n\}\) due successioni a termini reali \textit{positivi}, tali che \(a_n \sim b_n\).\\
Allora \(\sum a_n\) e \(\sum b_n\) hanno lo stesso carattere (entrambe convergono, entrambe divergono, etc).\\
\\
Solitamente si applica per i limiti notevoli.

\subsubsection{Criterio del rapporto}
\[\lim_{n \to +\infty} \frac{a_{n+1}}{a_n} =
\begin{cases}
L < 1 \quad \implies \quad \sum a_n \neq \infty\\
L > 1 \quad \implies \quad \sum a_n = \infty\\
L = 1 \quad \implies \quad unknown
\end{cases}\]

\subsubsection{Criterio della radice}
Sia \(\{a_n\}_{n \in \mathbb{N}^+}\).\\
Supponiamo che \(\lim_{n \to +\infty} \sqrt{a_n}^n = L\).
Allora...
\[\begin{cases}
L < 1 \quad \implies \quad \sum a_n convergente\\
L > 1 \quad \implies \quad \sum a_n divergente\\
L = 1 \quad \quad unknown
\end{cases}\]

\subsection{Serie a termini qualunque}

\subsubsection{Criterio di Leibniz}
\[\sum^{\infty}_{n=0} (-1)^n a_n\]
Se:
\[\begin{cases}
a_n \geq 0
a_n \geq a_{n+1}
a_n \to 0
\end{cases}\]

\subsubsection{Criterio di convergenza assoluta}
Se:
\[\sum^{\infty}_{n=0} |a_n| = \infty\]
Allora:
\[\sum^{\infty}_{n=0} a_n = \infty\]

\subsection{Dimostrazione dei criteri}
\subsubsection{Criterio del confronto}

\[S_n = \sum^n_{k=1} a_k\]
\[S_n^* = \sum^n_{k=1} b_k\]

\subsubsection{Criterio del confronto asintotico}
\[\lim_{n \to +\infty} \frac{a_n}{b_n} = 1\]
Usiamo la definizione di limite:
\[\forall \epsilon > 0 \exists n' : \forall n \geq n', 1 - \epsilon \leq \frac{a_n}{b_n} \leq 1 + \epsilon\]
\[b_n * (1 - \epsilon) \leq \frac{a_n}{b_n} \leq b_n * (1 + \epsilon)\]
Ho ora un'espressione a cui è applicabile il criterio del confronto.

Per la proprietà di monotonia:
\[0 \leq a_k \leq b_k \quad \implies \quad 0 \leq S_n \leq S_n^*\]

\subsubsection{Criterio della radice}
\[\forall \epsilon > 0, \exists n' : \forall n \geq n', L - \frac{\epsilon}{2} \leq \sqrt{a_n}^n \leq L + \frac{\epsilon}{2}\]
Per il funzionamento stesso della radice:
\[L < 1 \implies \exists \epsilon > 0 : L + \epsilon < 1; L < 1 - \epsilon\]
Dunque...
\[\sqrt{a_n}^n \leq 1 - \epsilon + \frac{\epsilon}{2} = 1 - \frac{\epsilon}{2}\]
Ho finalmente raggiunto un punto in cui posso usare il criterio del confronto:
\[\sum a_n \leq \sum (1 - \frac{\epsilon}{2})^2\]

\section{Tipi di esercizi}
Gli esercizi con le serie principalmente sono di tre tipi: calcolare la somma (il valore) di una serie, studiare la convergenza di una serie e studiare la convergenza di una serie che varia in base a un parametro.\\

\subsection{Serie geometriche}

\[\sum^{\infty}_{n=0}q^n\qquad se |q| < 1 \quad = \frac{1}{1-q}\]

\subsubsection{Esempio serie geometrica}
Calcolare \(\sum^{\infty}_{n=0}\frac{1}{2^n}\).\\

\subsection{Serie armonica generalizzata}
\[\sum^{\infty}_{n=0} \frac{1}{n^\alpha} \quad
\begin{cases}
\neq \infty \quad se \quad \alpha > 1\\
= \infty \quad se \quad \alpha \leq 1
\end{cases}
\]

\paragraph{Svolgimento}
E' una serie geometrica di ragione \(\frac{1}{2}\), quindi la somma vale \(\frac{1}{1-\frac{1}{2}} = 2\).

\subsubsection{Serie geometrica nascosta}
Calcolare \(\sum^{\infty}_{n=0}\frac{2^{n-1}}{3^n}\).\\

\paragraph{Svolgimento}
C'è una serie geometrica nascosta: è possibile convertire la somma in \(\frac{1}{2} \sum^{\infty}_{n=0}(\frac{2}{3})^n\), che è una serie geometrica di ragione \(\frac{2}{3}\).\\
Dunque, la somma vale \(\frac{1}{1-\frac{2}{3}} = \frac{3}{2}\).

\subsubsection{Serie geometrica con inizio spostato}
Calcolare \(\sum^{\infty}_{n=1}(\log(3) - 1)^n\).

\paragraph{Svolgimento}
Verifichiamo che la ragione sia effettivamente \(< 1\): \(log(3) - 1 < 1\) è vero.\\
Si converte la serie in \(\sum^{\infty}_{n=0}((\log(3) - 1)^n) - 1\).\\
E' diventata una serie geometrica di ragione \(\log(3) - 1\) a cui dovrà essere sottratto 1 dal risultato finale.

\subsection{Condizione necessaria}
Studia la convergenza di \(\sum^{\infty}_{n=1}(1 + \frac{1}{n!})^n\).

\paragraph{Svolgimento}
\[\sum^{\infty}_{n=1} (e^{n log(1 + \frac{1}{n!})})\]
\[\sum^{\infty}_{n=1} (e^{n \frac{1}{n!})})\]
\[e^{n \frac{1}{n!}} \to 1\]
Dato che l'argomento delle serie non è infinitesimo, allora possiamo dire che la serie non converge.

\subsection{Dipendenti da parametro}
Calcolare per quali valori di x la serie seguente converge.
\[\sum^{\infty}_{n=1} (\frac{x-2}{4})^n\]

\paragraph{Svolgimento}
Riconosciamo che è una serie geometrica, e sappiamo che converge se la sua ragione è \(|r| < 1\).\\
Calcoliamo per quali valori è presente quella ragione:
\[| \frac{x-2}{4} | < 1\]
\[-1 < \frac{x-2}{4} | < 1\]
\[-2 < x < 6\]

\subsection{Criterio del confronto difficile}
\[\sum^{\infty}_{n=1} \frac{\log^2 n}{n \sqrt{n}}\]
\[\lim_{n \to +\infty} \frac{\log^2 n}{n \sqrt{n}} = 0\]
Non concludo nulla da questo limite; devo usare un criterio.
\[\sum^{\infty}_{n=1} \frac{\log^2 n}{n \sqrt{n}}\]
\[\sum^{\infty}_{n=1} \frac{\log^2 n}{n^{\frac{3}{2}}}\]
\[\lim_{n \to \infty} \frac{log n}{n^\alpha} = 0\]
\[\log n \leq n^\alpha\]
\[\log n \leq n^\frac{1}{8}\]
\[\log^2 n \leq n^\frac{1}{4}\]
\[\frac{log^2 n}{n \sqrt{n}} \leq \frac{n^\frac{1}{4}}{n \sqrt{n}} = \frac{1}{n^{\frac{5}{4}}}\]
Applichiamo poi il teorema di confronto.
[TBD]

\subsection{Criterio di confronto asintotico difficile}
Determinare per quali valori di \(\alpha > 0\) la serie converge.
\[\sum^\infty_{n=1} \frac{1 + e^{-n}}{\sqrt{n^\alpha} + log n}\]
Non posso usare la condizione necessaria, perchè \(a_n \to 0\).\\
Applico il criterio del confronto asintotico.
\[a_n \sim \frac{1}{n^\frac{\alpha}{2}}\]
E' una serie armonica generalizzata.\\
Per \(\alpha > 2\), la serie converge, mentre per \(\alpha \leq 2\) la serie diverge.

\subsection{Criterio della radice}
\[\sum_{n=1}^\infty \frac{e^{n^2}}{n^{2n}}\]
\[\sqrt{a_n}^n = (\frac{e^{n^2}}{n^{2n}}^{\frac{1}{n}}\]
\[= \frac{e^n}{n^2} \to +\infty\]

\subsection{Criterio del rapporto}
\[\sum^\infty_{n=1} \frac{e^{n^2}}{n^{2n}}\]
\[\frac{a_{n+1}}{a_n} \to L\]
\[\frac{e^{(n+1)^2}}{(n+1)^{2(n+1)}} * \frac{n^{2n}}{e^{n^2}}\]
\[\frac{e^{2n+1}}{(n+1)^2} * (\frac{n}{n+1})^2n\]
\[\frac{e^{2n+1}}{(n+1)^2} * (\frac{1}{1+\frac{1}{n})^2n}\]
\[\lim_{n \to \infty} \frac{e^{2n+1}}{(n+1)^2} * (\frac{1}{1+\frac{1}{n})^2n}\]
\[+\infty * \frac{1}{e} = +\infty\]
\(+\infty > 1\), dunque la serie diverge.

\end{document}