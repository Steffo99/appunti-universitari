\documentclass{article}
\usepackage[utf8]{inputenc}
\usepackage{mathtools}
\usepackage{amssymb}
\usepackage{centernot}
% New symbols
\let\oldsqrt\sqrt
\def\sqrt{\mathpalette\DHLhksqrt}
\def\DHLhksqrt#1#2{%
\setbox0=\hbox{$#1\oldsqrt{#2\,}$}\dimen0=\ht0
\advance\dimen0-0.2\ht0
\setbox2=\hbox{\vrule height\ht0 depth -\dimen0}
{\box0\lower0.4pt\box2}}
% End new symbols
\begin{document}

\section{Equazioni di numeri complessi}
Come possiamo fare a risolvere equazioni in numeri complessi?\\
Una possibile soluzione è quella di applicare la definizione di numero complesso \(z = a + \i b\).\\
Effettuiamo le seguenti sostituzioni:
\[Re z = a\]
\[Im z = b\]
\[z = a + \i b\]
\[zsegnato = a - \i b\]
Probabilmente giungeremo a un risultato \(= 0 + 0i\); 

\end{document}
