\documentclass{article}
\usepackage[utf8]{inputenc}
\usepackage{mathtools}
\usepackage{amssymb}
% New symbols
\let\oldsqrt\sqrt
\def\sqrt{\mathpalette\DHLhksqrt}
\def\DHLhksqrt#1#2{%
\setbox0=\hbox{$#1\oldsqrt{#2\,}$}\dimen0=\ht0
\advance\dimen0-0.2\ht0
\setbox2=\hbox{\vrule height\ht0 depth -\dimen0}
{\box0\lower0.4pt\box2}}
% End new symbols

\begin{document}
\section{Definizione topologica di limite}
\[\lim_{x \to x_0} f(x) = l\]
\[\forall U_l \exists V_{x_0} : \forall x \neq x_0, (x \in V_{x_0} \implies f(x) \in U_l)\]

\section{Limite finito al finito}
\[\lim_{x \to x_0} f(x) = l \Leftrightarrow\]
\[\forall \epsilon > 0, \exists \delta > 0 : \forall x \neq x_0, |x - x_0| < \delta\]
Se un limite esiste, e in un certo punto il suo limite è uguale al valore del punto, allora \(f\) è \textbf{continua} in quel punto.

\section{Funzioni continue}
Sia \(f : \subseteq \mathbb{R} \to \mathbb{R}\), e sia \(x_0\) un \textit{punto di accumulazione} per il dominio \(D\) della funzione, appartenente al dominio della funzione.\\
f(x) è \textbf{continua} in \(x_0\) se:
\[\lim_{x \to x_0} f(x) = f(x_0)\]
Diciamo che è continua in generale se la formula superiore è vera \(\forall x \in D\).\\
La continuità è infatti un concetto locale: i valori esterni al dominio sono ignorati.

\subsection{Esempio}
\[f(x) =
\begin{cases}
    1\quad se \quad x \geq 0\\
    0\quad se \quad x < 0
\end{cases}\]
\[\lim_{x \to 1} f(x) = f(1) = 1\]
In 1, \(f\) è continua, perchè il suo limite esiste ed è uguale a 1.
\[\nexists \lim_{x \to 0} f(x)\]
In 0, \(f\) non è continua, perchè il suo limite non esiste.

\subsection{Esempio}
\[\lim_{x \to x_0} f(x)\]
\[f(x) =
\begin{cases}
    1\quad se \quad x \neq 0\\
    0\quad se \quad x = 0
\end{cases}\]
\[\lim_{x \to 0} f(x) = 1\]
In 0, \(f\) non è continua, perchè il suo limite esiste, ma è diverso da 0,

\subsection{Esempio}
\[f(x) = \frac{1}{x}\]
E' una funzione continua? Sì, perchè è continua per tutti i punti del suo dominio. 0, infatti, non è nel suo dominio.

\section{Definizione successionale di limite}
La \textit{definizione topologica di limite} è equivalente alla seguente definizione:
\[\lim_{x \to x_0} f(x) = l\]
\[\Updownarrow\]
\[\forall \{x_n\}_{n \neq 0 \in \mathbb{N}}; (x_n \to x_0) \implies f(x_n) \to l\]

\section{Funzioni asintotiche}
Si dice che due funzioni sono \textbf{asintotiche} per \(x \to x_0\) se:
\[\lim_{x \to x_0} \frac{f(x)}{g(x)} = 1\]
Dunque, si dice che \(f\) è asintotico a \(g\) in \(x_0\):
\[\sin x \sim x \qquad x \to 0\]

\section{Limiti notevoli}
\subsection{Seno di x su x}
\[\lim_{x \to 0} \frac{\sin x}{x} = 1\]
\[\sin x \sim x \qquad x \to 0\]

\subsubsection{Esempio}
\[sin(n) \not\sim n \qquad x \to +\infty\]
\[lim_{x \to +\infty} \frac{sin n}{n} = 0\]

\subsection{Tangente di x su x}
\[\lim_{x \to 0} \frac{\tan x}{x} = 1\]
\[\tan x \sim x \qquad x \to 0\]

\subsection{Arcotangente di x su x}
\[\lim_{x \to 0} \frac{\arctan x}{x} = 1\]
\[\arctan x \sim x \qquad x \to 0\]

\subsubsection{Esempio}
\[\lim_{n \to +\infty} \arctan \frac{1}{n^2} = 0\]
\[\arctan \frac{1}{n^2} \sim \frac{1}{n^2} \qquad n \to +\infty\]

\subsubsection{Esempio}
\[\lim_{x \to +\infty} \frac{\arctan n^2}{n^2} = 0\]
\(n^2 \to +\infty\), non tende a 0, quindi non possiamo applicare il limite notevole.

\subsection{Quello che fa un mezzo}
\[\lim_{x \to 0} \frac{1 - cos x}{x^2} = \frac{1}{2}\]
\[\lim_{x \to 0} \frac{1 - cos x}{\frac{1}{2} x^2} = 1\]
\[(1 - cos x) \sim (\frac{1}{2} x^2) \qquad x \to 0\]

\subsection{Naturale}
\[\lim_{x \to 0} \frac{e^x - 1}{x} = 1\]
\[(e^x - 1) \sim x \qquad x \to 0\]

\subsubsection{Esempio}
\[\lim_{n \to +\infty} \frac{e^\frac{1}{n} - 1}{\frac{1}{n}} = 1\]
L'argomento \(\frac{1}{n}\), per n che tende a più infinito, tende a 0; pertanto, possiamo applicare il limite notevole.

\subsubsection{Esempio}
\[\lim_{n \to +\infty} \frac{e^{\frac{1}{\sqrt{n}}} - 1}{\frac{1}{\sqrt{n}}} = 1\]

\subsubsection{Esempio}
\[\lim_{n \to +\infty} \frac{e^{\frac{1}{\sqrt{n}}}}{\frac{1}{\sqrt{n}}} = +\infty\]

\subsubsection{Esempio}
\[\lim_{n \to \infty} \frac{e^{\frac{1}{x}}}{\frac{1}{x}} \nexists\]
Questo limite non esiste, perchè per \(n \to +\infty\) vale \(+\infty\), mentre per \(n \to -\infty\) vale \(0\).

\subsubsection{Esempio}
\[\lim_{n \to +\infty} \frac{e^{\frac{1}{n^2}} - 1}{\frac{1}{n}} =\]
\[\lim_{n \to +\infty} (\frac{e^{\frac{1}{n^2}} - 1}{\frac{1}{n^2}} * \frac{1}{n}) = 0\]

\subsubsection{Altri esempi}
Non avevo voglia di scriverli, quindi li ho omessi.

\subsection{Risulta e}
\[\lim_{x \to +\infty} (1 + \frac{1}{x})^x\]

\subsection{Logaritmico}
\[\lim_{x \to 0} \frac{\log(1+x)}{x} = 1\]
\[\log(1+x) \sim x \qquad x \to 0\]

\subsubsection{Esempio}
\[\lim_{x \to +\infty} \frac{\log(1+x)}{x} = 0\]

\subsection{L'ultima}
\[\lim_{x \to 0^+} x \log x = 0\]

\subsubsection{Esempio}
\[\lim_{x \to 0^+} x^x = e^(x \log x) = 1\]
\[\lim_{x \to +\infty} x^x = e^(x \log x) = +\infty\]

\section{Esempi}
\subsubsection{Esempio}
\[\lim_{x \to +\infty} \frac{\sin x + 2x^2 + e^{-x}}{2x - 2x^2 + e^{-x}} = -1\]
Prevale \(2x^2\), perchè \(e^{-x}\) tende a 0.

\subsubsection{Esempio}
\[\lim_{x \to 0} \frac{\sin x + 2x^2 + e^{-x}}{2x - 2x^2 + e^{-x}} = 0\]
Prevale \(e^{-x}\), perchè è l'unico che non tende a 0, tendendo invece a 1.

\subsubsection{Esempio}
\[\lim_{x \to 0} \frac{| x - \pi |}{x - \pi} = -1\]
Perchè il valore assoluto diventa \(\pi - x\), e dopo prevale \(-x\).

\subsubsection{Esempio}
\[\lim_{x \to \pi} \frac{| x - \pi |}{x - \pi} = +-1\]
Dipende da che direzione ci approcciamo a \(\pi\): per \(x \to \pi^+\), il limite vale 1, ma per \(x \to \pi^-\), il limite vale -1.

\subsubsection{Esempio}
\[\lim_{x \to \pi} \frac{\cos x + 1}{(x - \pi)^2} = [0/0]\]
\[z = x - \pi\]
\[\lim_{z \to 0} \frac{\cos (z + \pi) + 1}{(z + \pi - \pi)^2}\]
\[\lim_{z \to 0} \frac{\cos z \cos \pi - \sin z \sin \pi + 1}{z^2}\]
\[\lim_{z \to 0} \frac{-\cos z + 1}{z^2} = \frac{1}{2}\]
Applichiamo il cambio di variabile in modo di avere un limite per 0: dopo, applichiamo la formula del coseno della somma.

\subsubsection{Esempio}
\[\lim_{x \to 0} \frac{log(1+x) - sin x}{x + sin x} = [0/0]\]
\[\lim_{x \to 0} \frac{log(1+x)}{x + sin x} - \frac{sin x}{x + sin x}\]
\[\lim_{x \to 0} \frac{log(1+x)}{x (1 + \frac{sin x}{x})} - \frac{sin x}{x (1 + \frac{sin x}{x})}\]
\[\frac{1}{1 + 1} - \frac{1}{1 + 1} = 0\]
Separiamo il limite in due: è un'operazione che funziona solo se nessuno dei due nuovi limiti risulta infinito o indeterminato.

\subsubsection{Esempio}
\[\lim_{x \to 0} \frac{\cos x - 2^x}{\arctan(\log(\sin \sqrt{x} + 1))} = [0/0]\]
\[\lim_{x \to 0} \frac{\cos x - x^2 + 1 - 1}{\arctan(\log(\sin \sqrt{x} + 1))}\]
Per \(x \to 0\), \(\sin \sqrt{x} \sim \sqrt{x}\); \(\log (1 + z) \sim z\); \(\arctan z \sim z\), dunque.
\[\lim_{x \to 0} \frac{\cos x - x^2 + 1 - 1}{\sqrt{x}}\]
\[\lim_{x \to 0} \frac{\cos x - 1}{\sqrt{x} - \lim_{x \to 0}\frac{x^2} + 1 - 1}{\sqrt{x}}\]
\[\lim_{x \to 0} \frac{-\frac{1}{2}x^2}{\sqrt{x}} - \lim_{x \to 0}\frac{x^2 - 1}{\sqrt{x}}\]
\[0 - 0 = 0\]

\end{document}
