\documentclass{article}
\usepackage[utf8]{inputenc}
\usepackage{mathtools}
\usepackage{amssymb}
\usepackage{centernot}
% New symbols
\let\oldsqrt\sqrt
\def\sqrt{\mathpalette\DHLhksqrt}
\def\DHLhksqrt#1#2{%
\setbox0=\hbox{$#1\oldsqrt{#2\,}$}\dimen0=\ht0
\advance\dimen0-0.2\ht0
\setbox2=\hbox{\vrule height\ht0 depth -\dimen0}
{\box0\lower0.4pt\box2}}
\newcommand{\iu}{\mathrm{i}\mkern1mu}
% End new symbols

\begin{document}

\section{Integrale}

\[f : [a, b] \to \mathbb{R} \qquad f\ limitata\]

Definiamo \textbf{suddivisione} o partizione di \([a, b]\):
\[\mathcal{A} = \{ x_0, x_1, ..., x_n \}\]
tale che:
\[a = x_0 < x_1 < ... < x_n = b\]
\\
Sono praticamente tanti pezzetti dell'intervallo \([a, b]\).\\

\subsection{Primo modo}
Prendiamo una suddivisione equispaziata \(h = \frac{b - a}{n}\): dunque, gli intervallini saranno \(x_j = a + jh\) per \(j = 0, 1, ..., n\).\\
\(x_n = a + nh = a + n \frac{b - a}{n} = b\)

[VSC crashed.]\\

\subsection{Secondo modo}

[VSC crashed.]\\

f è integrabile se \(s(f)\) (integrale inferiore) \(= S(f)\) (integrale superiore).

\section{Classi di funzioni integrabili}
\[f : [a, b] \to \mathbb{R}\ limitata\]
Allora:
\(f\ integrabile\ su\ [a, b] \implies \forall \epsilon > 0, \exists \mathcal{A}\ suddivisione\ di\ [a, b]\ : S(f, \mathcal{A}) - s(f, \mathcal{A}) \leq \epsilon\)

\paragraph{Dimostrazione}
Per semplicità, \(\mathcal{A}\) è equispaziata.\\
\(S(f, \mathcal{A}_n) = \sum_{j=1}^n (x_j - x_{j-1}) sup(f) = \frac{b - a}{n} \sum_{j=1}^n sup(f) = \frac{b - a}{n} \sum_{j=1}^n f(x_j)\)\\
\(s(f, \mathcal{A}) = \sum_{j=1}^n (x_j - x_{j-1}) inf(f) = \frac{b - a}{n} \sum_{j=1}^n sup(f) = \frac{b - a}{n} \sum_{j=1}^n f(x_{j-1})\)\\
Effettuiamo un cambio di variabile: \(j-1 = i\), e poi rinominiamolo nuovamente in \(j\).\\
\(s(f, \mathcal{A}) = \sum_{j=1}^n (x_j - x_{j-1}) inf(f) = \frac{b - a}{n} \sum_{j=1}^n sup(f) = \frac{b - a}{n} \sum_{j=1}^n f(x_{j-1}) = \frac{b - a}{n} \sum^{n-1}_{i=0} f(x_i) = \frac{b - a}{n} \sum_{j = 0}^{n - 1} f(x_m) = \frac{b - a}{n} \sum_{j = 0}^{n - 1} = \frac{b - a}{n} ( \sum_{j = 0}^{n - 1} f(x_m) + f(x_0) )\)\\
\(S(f, \mathcal{A}) - s(f, \mathcal{A}) = \frac{b - a}{n} (f(x_n) - f(x_0)) = \frac{b - a}{n} (f(b) - f(a))\)\\
\\
Potrebbe essere da ripassare bene.

\section{Teorema}
\[f : [a, b] \to \mathbb{R}\ continua \implies integrabile\ in\ [a, b]\]

\section{Teorema}
\[\int_a^b f(x) dx = 0\]
\[f\ continua\ in\ [a, b] \implies f = 0\ in\ [a, b]\]

\paragraph{Dimostrazione}
Supponiamo per assurdo che \(\exists x_0 \in [a, b]\).
Allora, \(f(x_0) > 0\), e per la proprietà di continuità, \(\exists [a', b'] : \forall x \in [a', b'], \int_a^b f(x) dx \neq 0\).

\end{document}

