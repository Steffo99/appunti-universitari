\documentclass{article}
\usepackage[utf8]{inputenc}
\usepackage{mathtools}
\usepackage{amssymb}
\usepackage{centernot}
% New symbols
\let\oldsqrt\sqrt
\def\sqrt{\mathpalette\DHLhksqrt}
\def\DHLhksqrt#1#2{%
\setbox0=\hbox{$#1\oldsqrt{#2\,}$}\dimen0=\ht0
\advance\dimen0-0.2\ht0
\setbox2=\hbox{\vrule height\ht0 depth -\dimen0}
{\box0\lower0.4pt\box2}}
\newcommand{\iu}{\mathrm{i}\mkern1mu}
% End new symbols
\begin{document}

\section{Teorema di Weierstrass}

\paragraph{Ipotesi}
\footnotesize Devo scrivere per forza qualcosa qua.\\\normalsize
\([a, b]\) intervallo \textit{chiuso} e \textit{limitato}\\
\(f\) continua su \([a, b]\)

\paragraph{Tesi}
\(f\) ha \textit{massimo} e \textit{minimo} su \([ a, b ]\)\\
\[\exists x_M, \exists x_n : f(x_M) \leq f(x) \leq f(x_n)\]

\paragraph{Tabella lettere}
\footnotesize Per capirci qualcosa in più.\\\normalsize
\(f\) \quad funzione\\
\(M\) \quad reale, estremo superiore della funzione\\
\(x_M\) \quad reale, punto in cui la funzione raggiunge il valore di \(M\)\\
\(x_n\) \quad successione, ???\\
\(y_n\) \quad successione, ???

\paragraph{Dimostrazione}
\footnotesize Dimostrazione per il minimo omessa, in quanto opposta di questa.\\\\\normalsize
Sia \(M = sup(f) = sup \{f(x) : x \in [a, b]\}\).\\\\
Devo dimostrare che M venga raggiunto in almeno un punto della funzione: \(\exists x_M \in [a, b]\) tale che \(f(x_M) = M\).\\\\
\(M\) è il minimo dei maggioranti; se considero un qualsiasi numero \(y_n < M\), questo \textit{non è un maggiorante} per la definizione di estremo superiore.\\\\
Allora, creo una successione \(x_n\) in modo che \(y_n < f(x_n) \leq M\).\\\\
Dato che \(y_n\) tende ad \(M\), per il \textsc{Teorema dei Carabinieri} \(f(x_n) \to M\).\\\\
Il fatto che \(x_n\) sia \(\in [a, b]\) ci fa dire che la successione sia \textit{limitata}.\\\\
Essendo limitata, per il \textsc{Teorema di Bolzano-Weierstrass} possiamo estrarre sicuramente una sottosuccessione \(x_{n_k}\) tale che essa tenda a un valore finito \(\to x_M\).\\\\
Essendo \(f\) una funzione continua, allora \(f(x_{n_k} \to f(x_n)\).\\\\
Dato che tutte le sottosuccessioni estratte tendono allo stesso valore, allora possiamo dire che \(M = f(x_M)\).

\end{document}