\documentclass{article}
\usepackage[utf8]{inputenc}
\usepackage{mathtools}
\usepackage{amssymb}
% New root
\let\oldsqrt\sqrt
\def\sqrt{\mathpalette\DHLhksqrt}
\def\DHLhksqrt#1#2{%
\setbox0=\hbox{$#1\oldsqrt{#2\,}$}\dimen0=\ht0
\advance\dimen0-0.2\ht0
\setbox2=\hbox{\vrule height\ht0 depth -\dimen0}%
{\box0\lower0.4pt\box2}}
% End new root

\begin{document}

\section{Intorno}
Si dice \textbf{intorno} di un numero \(x_0 \in \mathbb{R}\) un qualsiasi intervallo aperto del tipo:\\
\((x_0 - \delta, x_0 + \delta)\)

\section{Punti isolati}
Si dice \textbf{punto isolato} un numero \(x_0 \in \mathbb{R}\) se esiste un intorno \(U\) di \(x_0\) tale che al suo interno sia presente solo il punto stesso e nient'altro.
\(U \bigcup A = {x_0}\)

\section{Punti di accumulazione}
Si dice \textbf{punto di accumulazione} un punto tale che non possano esistere suoi intorni che includano il punto stesso come unico elemento.\\
Sono gli opposti dei punti isolati.
\(U \bigcap A \ {X_0} \neq \emptyset\).

\section{Definizione topologica di limite}
[todo]\\\\
\(\forall U_l\) intorno di l\\
\(\exists V_{x_0}\) intorno di \(x_0\)\\
\(: x \in V_{x_0}\)\\
\(x \neq x_0 \implies f(x) \in U_l\)

\section{Limite finito all'infinito}
\(\lim_{x \to +\infty} f(x) = l \in \mathbb{R}\)\\
\(\forall \epsilon > 0, \exists K > 0 : \forall x > K, l - \epsilon < f(x) < l + \epsilon\)\\\\
\(\lim_{x \to -\infty} f(x) = l \in \mathbb{R}\)\\
\(\forall \epsilon > 0, \exists K > 0 : \forall x < K, l - \epsilon < f(x) < l + \epsilon\)\\\\
\(\epsilon\) è l'ampiezza della "striscia" in cui stanno i valori di f(x), \(K\) è la "barriera" dei valori della x e \(l\) è il valore del limite, ovvero "l'altezza" a cui si trova la striscia.

\section{Limite infinito all'infinito}
\[\lim_{x \to +\infty} f(x) = +\infty \in \mathbb{R}\]
\[\forall H > 0, \exists K > 0 : \forall x > K, f(x) > H\]\\
\[\lim_{x \to -\infty} f(x) = +\infty \in \mathbb{R}\]
\[\forall H > 0, \exists K < 0 : \forall x < K, f(x) > H\] \\
\[\lim_{x \to +\infty} f(x) = -\infty \in \mathbb{R}\]
\[\forall H < 0, \exists K > 0 : \forall x > K, f(x) < H\]
\[\lim_{x \to -\infty} f(x) = -\infty \in \mathbb{R}\]
\[\forall H < 0, \exists K < 0 : \forall x < K, f(x) < H\]

\section{Asintoto obliquo}
La funzione f ha la retta \(y = mx + q\) come \textbf{asintoto obliquo} per \(x \to +\infty\) se:
\[\lim_{x \to +\infty} (f(x) - mx - q) = 0\]

f ha asintoto obliquo se e solo se:
\[\lim_{x \to +\infty} \frac{f(x)}{x} = m \neq 0 \neq \infty\]
\[\lim_{x \to +\infty} (f(x) - mx) = q \neq \infty\]
\\
Esempio:
\(f(x) = e^x + 2x + 1\) ha asintoto obliquo se \(x \to -\infty\):\\
\[\lim_{x \to -\infty} \frac{e^x + 2x + 1}{x} = 2\]
\[\lim_{x \to -\infty} (e^x + 2x + 1 - 2x) = 1\]

\section{Limite infinito al finito}
\[\lim_{x \to x_0} f(x) = +\infty\]
\[\forall H > 0, \exists \delta > 0 : \forall x \neq x_0, | x - x_0 | < \delta \implies f(x) > H\]
\[\lim_{x \to x_0} f(x) = -\infty\]
\[\forall H < 0, \exists \delta > 0 : \forall x \neq x_0, | x - x_0 | < \delta \implies f(x) < H\]

\section{Asintoto verticale}
\[\lim_{x \to 0^+} \log x = -\infty\]

\end{document}

