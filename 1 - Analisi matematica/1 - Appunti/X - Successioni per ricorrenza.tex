\documentclass{article}
\usepackage[utf8]{inputenc}
\usepackage{mathtools}
\usepackage{amssymb}
% New root
\let\oldsqrt\sqrt
\def\sqrt{\mathpalette\DHLhksqrt}
\def\DHLhksqrt#1#2{%
\setbox0=\hbox{$#1\oldsqrt{#2\,}$}\dimen0=\ht0
\advance\dimen0-0.2\ht0
\setbox2=\hbox{\vrule height\ht0 depth -\dimen0}%
{\box0\lower0.4pt\box2}}
% End new root

\begin{document}
\section{Successioni per ricorrenza}

Una successione per ricorrenza è una successione definita stabilendo l'elemento di partenza \(a_0\) e l'espressione per il valore successivo \(a_{n+1}\).\\
E' sempre definita su una semiretta dei numeri naturali: non può esistere un valore per cui non è definito un elemento ma è definito il suo successivo.\\\\

\(
	\begin{cases}
	a_0 = \alpha\\
	a_{n+1} = f(a_n)
	\end{cases}
\)\\\\

\section{Successioni per ricorrenza monotone}
Una successione per ricorrenza è monotona se il suo risultato non ha mai punti critici, ovvero \(\)

\subsection{Esercizio}
\paragraph{Ipotesi}
\(\begin{cases}
	a_0 = \alpha\\
	a_{n+1} = \sqrt{a_n} + 100
\end{cases}\)\\\\

\paragraph{Tesi}
\(\forall n \in \mathbb{N}, a_n \geq 0\)

\paragraph{Dimostrazione}
Inizio dell'induzione:\\
Se \(\alpha \geq 0 \), allora \(\alpha \in S\)\\\\
Passo induttivo:\\
\(n \in S \implies n+1 \in S \)\\
\(a_n \geq 0 \implies a_{n+1} \geq 0\)\\
\(\sqrt{a_n} + 100 \geq 0\)\\
\(\sqrt{\alpha} \geq -100\)\\
\(\alpha \geq 0\)

\subsection{Esercizio}
\paragraph{Ipotesi}
\(\begin{cases}
	a_0 = \alpha\\
	a_{n+1} = \frac{a_n}{2} + 3
\end{cases}\)

\paragraph{Tesi}
Limite della ricorrenza.

\paragraph{Svolgimento}
\(a_0 = \alpha\)\\
\(a_1 = \frac{a_0}{2} + 3 = \frac{\alpha}{2} + 3\)\\
\(a_2 = \frac{\frac{\alpha}{2} + 3}{2} + 3 = \frac{\alpha}{4} + \frac{3}{2} + 3\)\\
\(a_n = \frac{\alpha}{2^n} + 3 * (1 + \frac{1}{2} + \frac{1}{4} + \dots + \frac{1}{2^{n-1}})\)\\
\(a_n = \frac{\alpha}{2^n} + 3 * \displaystyle\sum_{i=\alpha}^{n-1} \frac{1}{2^i}\)\\
Congetturo il valore della somma:\\
[svolgimento omesso]\\
\(= 2(1 - \frac{1}{2^n})\)\\
Ora posso calcolare il limite:\\
[svolgimento omesso]\\
\(= 6\)\\

\section{Un modo più veloce}
\(a_n \to l\)\\
\(a_{n+1} \to l\)\\
\(a_{n+1} = \frac{a_n}{2} + 3\)\\
\(\frac{a_n}{2} \to \frac{l}{2}\)\\
\(l = \frac{l}{2} + 3\)\\
\(l = 6\)\\\\

Ma \(a_n\) ha veramente limite? Se è \textbf{monotona}, ha sempre un limite.\\
Se non ha limite, non possiamo usare questo metodo, perchè darà come risultato \(\frac{\infty}{\infty}\), una forma di indecisione.

\subsection{Esempio}
\(\begin{cases}
	a_0 = \alpha\\
	a_{n+1} = \sqrt{a_n + 2}
\end{cases}\)

\paragraph{Svolgimento}
\(\begin{cases}
	y = x\\
	y = \sqrt{x + 2}
\end{cases}\)\\

\(\begin{cases}
	y = x\\
	x = \sqrt{x + 2}
\end{cases}\)\\

\(\begin{cases}
	y = x\\
	x^2 = x + 2
\end{cases}\)\\

\(\begin{cases}
	y = x\\
	x^2 - x - 2 = 0
\end{cases}\)\\

\(\begin{cases}
	y = 2\\
	x = 2
\end{cases}\)\\

\(l = 2\)

\subsection{Esercizio}
\(\begin{cases}
	a_0 = \alpha\\
	a_{n+1} = \sqrt{a_n + 2}
\end{cases}\)\\
Dimostrare per induzione che \(\alpha \geq 2 \implies \forall n, a_n \geq 2\).

\paragraph{Ipotesi}
\(S = \{n \in \mathbb{N} : a_n \geq 2\}\)

\paragraph{Tesi}
\(S = \mathbb{N}\)

\paragraph{Dimostrazione}
\subparagraph{Passo base}
\(
	a_0 \geq 2\\
	0 \in S
\)

\subparagraph{Passo induttivo}
\(
	n \in S \implies n+1 \in S\\
	a_n \geq 2 \implies a_{n+1} \geq 2\\
	a_n \geq 2 \implies \sqrt{a_n + 2} \geq 2\\
\)\\
Per ipotesi, \(a_n \geq 2\), quindi \(a_n + 2 \geq 2 + 2 = 4\) e allora \(\sqrt{a_n + 2} \geq \sqrt{4} = 2\).

\subparagraph{Passo monotono}
Perchè la successione sia monotona, dobbiamo verificare che \(a_n+1 \leq a_n\).\\

\(\sqrt{a_n + 2} \leq a_n\)\\
Arrivo alla soluzione \(a_n \leq -1 \bigcup a_n \geq 2\).\\
Tengo solo \(a_n \geq 2\). Gli altri non sono numeri naturali.

\subparagraph{Passo finale}
\(a_n\) monotona \(\implies a_n \to l\).\\
Dunque, esiste un limite, finito o infinito che sia.\\
[todo] 

\end{document}
