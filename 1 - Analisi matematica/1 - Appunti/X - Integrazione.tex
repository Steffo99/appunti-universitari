\documentclass{article}
\usepackage[utf8]{inputenc}
\usepackage{mathtools}
\usepackage{amssymb}
\usepackage{centernot}
% New symbols
\let\oldsqrt\sqrt
\def\sqrt{\mathpalette\DHLhksqrt}
\def\DHLhksqrt#1#2{
\setbox0=\hbox{$#1\oldsqrt{#2\,}$}\dimen0=\ht0
\advance\dimen0-0.2\ht0
\setbox2=\hbox{\vrule height\ht0 depth -\dimen0}
{\box0\lower0.4pt\box2}}
\newcommand{\iu}{\mathrm{i}\mkern1mu}
\DeclarePairedDelimiter\abs{\lvert}{\rvert}
\DeclarePairedDelimiter\norm{\lVert}{\rVert}
\makeatletter
\let\oldabs\abs
\def\abs{\@ifstar{\oldabs}{\oldabs*}}
\let\oldnorm\norm
\def\norm{\@ifstar{\oldnorm}{\oldnorm*}}
\makeatother
\newcommand*{\Value}{\frac{1}{2}x^2}
\newcommand{\intab}{\int_a^b}
% End new symbols
\begin{document}

\section{\(\delta < 0, denominatore II grado\)}
\[\int \frac{1}{x^2 + 4x + 9} dx\]
Osserviamo che \(\int \frac{1}{x^2 + 1} dx = \arctan x + c\).
Provo allora a costruire qualcosa di simile all'arcotangente.
\[\int \frac{1}{x^2 + 4x + 9} dx = \int \frac{1}{x^2 + 4x + 4 + 5} dx = 5 \int \frac{1}{\frac{(x + 2)^2}{5} + 1} = 5 \arctan (\frac{x + 2}{\sqrt{5}}) + c\]

\section{Integrale generalizzato}
Vogliamo ampliare la nostra definizione di integrale, applicandolo a una \(f\) non limitata.
\[\int_a^{b-\epsilon} f(x) dx\]
Ha senso; la funzione è limitata in \([a, b - \epsilon]\).\\
Allora, possiamo fare l'integrale \textbf{generalizzato} o improprio, se \textsc{esiste} ed è \textsc{finito}:
\[\lim_{\epsilon \to 0^+} \int_a^{b - \epsilon} f(x) dx = \int_a^b f(x) dx\]

\subsection{Esercizi}

\subsubsection{Uso di parametri}
Dire per quali valori del \textit{parametro} \(\alpha\)...
\[\int_0^1 \frac{1}{x^{\alpha}} dx\]
Per \(\alpha \leq 0\), si ha che \(\int_0^1 x^-\alpha dx\), e quindi è un integrale standard.\\
Per \(\alpha > 0\), si ha che \(\int_0^1 \frac{1}{x^\alpha} dx\).\\
C'è un problema in \(x = 0\); la funzione non è limitata! Usiamo allora la definizione di integrale generalizzato.\\

\subsubsection{Calcolo integrali generalizzati con la definizione}
\textit{Calcola} l'integrale...
\[\int_0^1 \frac{1}{x^\alpha} dx = \lim_{\epsilon \to 0^+} \int_\epsilon^1 \frac{1}{x^\alpha} dx\]

Trovo l'insieme delle sue primitive:
\[\int x^{-\alpha} dx = \begin{cases}
    \log \abs{x} + c \qquad \alpha = 1\\
    \frac{x^{1-\alpha}{1 - \alpha} + c \qquad \alpha \neq 1
\end{cases}\]

Infine, applico il teorema fondamentale del calcolo:\\
Per \(\alpha = 1\):
\[\lim_{\epsilon \to 0^+} \int_\epsilon^1 \frac{1}{x^\alpha} dx = [\log \abs{x}]^1_\epsilon = \log 1 - \log \epsilon = - \log \epsilon\]
Per \(\alpha \neq 1\):
[mi sa fatica scriverlo ma è uguale a sopra... credo]

\subsubsection{Uso dei criteri}
\textit{Studiare} l'integrabilità...


\end{document}