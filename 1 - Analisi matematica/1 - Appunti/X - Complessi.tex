\documentclass{article}
\usepackage[utf8]{inputenc}
\usepackage{mathtools}
\usepackage{amssymb}
\usepackage{centernot}
% New symbols
\let\oldsqrt\sqrt
\def\sqrt{\mathpalette\DHLhksqrt}
\def\DHLhksqrt#1#2{%
\setbox0=\hbox{$#1\oldsqrt{#2\,}$}\dimen0=\ht0
\advance\dimen0-0.2\ht0
\setbox2=\hbox{\vrule height\ht0 depth -\dimen0}
{\box0\lower0.4pt\box2}}
\newcommand{\iu}{\mathrm{i}\mkern1mu}
% End new symbols
\begin{document}

\section{Equazioni in \(\mathbb{C}\)}
Come possiamo fare a risolvere equazioni in numeri complessi?\\
Una possibile soluzione è quella di applicare la definizione di numero complesso \(z = a + \iu b\).\\
Effettuiamo le seguenti sostituzioni:
\[Re z = a\]
\[Im z = b\]
\[z = a + \iu b\]
\[\bar{z} = a - \iu b\]
Spostando tutti gli elementi al primo membro, giungeremo ad avere al secondo membro \(= 0 + 0i\); possiamo allora fare un sistema con la parte reale e la parte immaginaria del primo membro e risolverlo per a e b; infine, dovremo verificare manualmente tutte le soluzioni trovate in questo modo.

\subsection{Esempio}
[todo, non l'ho copiato]

\subsection{Altro esempio}
\[\begin{cases}
    z \bar{w} = \iu\\
    |z|^2 w + z = 1
\end{cases}\]
Passiamo la seconda equazione ai coniugati.
\[\bar{|z|^2 w + z = 1} = \bar{1} = 1\]
\[\bar{|z|^2} \bar{w} + \bar{z} = 1\]
Vado a ricavare \(z\) dalla prima equazione.\\
Se \(z \neq 0\), allora...
\[\bar{w} = \frac{\iu}{z}\]
E obbligatoriamente \(z \neq 0\), perchè altrimenti l'equazione non sarebbe verificata.\\
Sappiamo che il modulo \(|z|^2 = (Re z)^2 + (Im z)^2 = z \bar{z}\), dunque tornando alla seconda equazione:
\[\frac{z \bar{z} \iu}{z} + \bar{z} = 1\]
\[\bar{z} \iu + \bar{z} = 1\]
Risolvo la seconda equazione:
\[a\iu + a + b - b\iu - 1 = 0\]
\[\begin{cases}
    a + b - 1 = 0\\
    a = b
\end{cases}\]
\[a = b = \frac{1}{2}\]
\[z = \frac{1}{2} + \frac{\iu}{2}\]
\[\bar{w} = \frac{\iu}{z} = \frac{\iu}{\frac{1}{2} + \frac{\iu}{2}} = \frac{2 \iu + 2}{2} = 1 + \iu\]

\section{Forma trigonometrica dei numeri complessi}
Possiamo rappresentare i numeri complessi in un'altra forma, invece che quella algebrica.\\
Rappresentiamo un complesso composto da un modulo \(\rho\) e un argomento \(\theta\) corrispondente all'angolo formato da il semiasse positivo del piano cartesiano e la semiretta che congiunge z e l'origine.
\[\rho = \sqrt{a^2 + b^2}\]
\[\begin{cases}
    a = Re z = |z| \cos(\theta)\\
    b = Im z = |z| \sin(\theta)
\end{cases}\]
[esempi omessi tanto sono sulle dispense]

\section{Teorema}
Siano \(z = \rho (\cos(\theta) + \iu \sin(\theta))\) e \(w = r (\cos \phi + \iu \sin(\theta))\), allora:
\[zw = \rho r (\cos(\theta + \phi) + \iu \sin(\theta + \phi)\]
\[\frac{z}{w} = \frac{\rho}{r} (\cos(\theta - \phi) + \iu \sin(\theta - \phi))\]

\subsection{Potenza di un complesso}
\[z^n = \rho^n (\cos(n \rho) + \iu sin (n \rho))\]

\subsubsection{Esempio}
Calcolare \((1 + \iu)^{16}\).

\paragraph{Svolgimento}
Troviamo la forma trigonometrica: 
\[1 + \iu = \sqrt{2} (\cos(\frac{\pi}{4}) + \iu \sin({\pi}{4}))\]
\[(1 + \iu)^{16} = 2^4 (\cos(4 \pi) + \iu sin(4 \pi))\]

\subsubsection{Esempio}
\[i^{2018} = i^{504 * 4} * i^{2} = -1\]

\section{Radici ennesime di numeri complessi}
Sia \(w \in \mathbb{C}, w \neq 0\), allora esistono \(n\) radici ennesime complesse \(z_0, z_1, ..., z_{n-1}\) di \(w\), tali che:
\[z^n_i = w \qquad i=0, ..., n-1\]
Inoltre:
\[w = r (\cos(\phi) + \iu \sin(\phi))\]
\[z_K = \rho_K (\cos(\phi_K) + \iu \sin(\phi_K)) \qquad k = 0, ..., n-1\]
\[\rho_K = r^{\frac{1}{n}}\]
\[\phi_K = \frac{\phi}{n} + \frac{2 \pi K}{n}\]

\end{document}