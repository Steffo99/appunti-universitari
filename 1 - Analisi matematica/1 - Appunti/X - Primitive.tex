\documentclass{article}
\usepackage[utf8]{inputenc}
\usepackage{mathtools}
\usepackage{amssymb}
\usepackage{centernot}
% New symbols
\let\oldsqrt\sqrt
\def\sqrt{\mathpalette\DHLhksqrt}
\def\DHLhksqrt#1#2{
\setbox0=\hbox{$#1\oldsqrt{#2\,}$}\dimen0=\ht0
\advance\dimen0-0.2\ht0
\setbox2=\hbox{\vrule height\ht0 depth -\dimen0}
{\box0\lower0.4pt\box2}}
\newcommand{\iu}{\mathrm{i}\mkern1mu}
\DeclarePairedDelimiter\abs{\lvert}{\rvert}
\DeclarePairedDelimiter\norm{\lVert}{\rVert}
\makeatletter
\let\oldabs\abs
\def\abs{\@ifstar{\oldabs}{\oldabs*}}
\let\oldnorm\norm
\def\norm{\@ifstar{\oldnorm}{\oldnorm*}}
\makeatother
\newcommand*{\Value}{\frac{1}{2}x^2}
\newcommand{\intab}{\int_a^b}
% End new symbols

\begin{document}

\section{Proprietà dell'integrale}
Siano \(f, g : [a, b] \to \mathbb{R}\).\\
Siano \(\alpha, \beta, a, r, b \in \mathbb{R}\).\\
Allora:
\begin{itemize}
\item \(\alpha f + \beta g\) è integrabile:\\
      \[\int_a^b (\alpha f(x) + \beta g(x)) dx = \alpha \int_a^b f(x) dx + \beta \int_a^b g(x) dx\]
\item Se \(a \leq r \leq b\), allora f è integrabile su \([a, r]\) e \([r, b]\), e in particolare:
      \[\int_a^b f(x) dx = \int_a^r f(x) dx + \int_r^b f(x) dx\]
\item Se \(f \geq g\), allora \(\int_a^b f(x) dx \geq \int_a^b g(x) dx\).
\item Se \(f\) è integrabile in \([a, b]\), allora \(\abs{f}\) è integrabile (ma non il contrario!).
\end{itemize}

\section{Teorema della media integrale}
\subsection{Prima parte}
\paragraph{Ipotesi}
\(f\) integrabile su \([a, b]\)

\paragraph{Tesi}
\[inf f \leq \frac{1}{b - a} \int_a^b f(x) dx \leq sup f\]

\paragraph{Dimostrazione}
Sappiamo che \(inf f \leq f(x) \leq sup f\).\\
Per la 3a proprietà dell'integrale, allora:
\[\intab inf f dx \leq \intab f(x) dx \leq \intab sup f dx\]
Possiamo portare fuori le costanti per la 1a proprietà:
\[inf f \intab 1 dx \leq \intab f(x) dx \leq sup f \intab 1 dx\]
Allora:
\[inf f (b - a) \leq \intab f(x) dx \leq sup f (b - a)\]
E se \(b - a \neq 0\)...
\[inf f \leq \frac{1}{b - a} \int_a^b f(x) dx \leq sup f\]


\subsection{Seconda parte}

\paragraph{Ipotesi}
\begin{itemize}
\item \(inf f \leq \frac{1}{b - a} \int_a^b f(x) dx \leq sup f\)
\item \(f\) continua su \([a, b]\)
\end{itemize}

\paragraph{Tesi}
\(\exists z : \frac{1}{b - a} \int_a^b f(x) dx = f(z)\)

\paragraph{Dimostrazione}
Cambiamo forma alla tesi:
\[\exists z : \intab f(x) dx = f(z) * (b - a)\]
Se la funzione è continua in \([a, b]\), per il teorema di Weierstrass sappiamo che:
\[\exists x_m, x_M : min f = m = f(x_m) \leq f(x) \leq f(x_M) = M = max f\]
Per la prima ipotesi, allora:
\[min f = inf f \leq \frac{1}{b - a} \intab f(x) dx \leq sup f = max f\]
Essendoci un minimo e un massimo, ed essendo la funzione continua, possiamo dire per il teorema dei valori intermedi che:
\[\exists z : \frac{1}{b - a} \int_a^b f(x) dx = f(z)\]

\section{Funzione primitiva}
Sia \(f : [a, b] \to \mathbb{R}\).\\
Si dice che \(G\) è \textbf{primitiva} di \(f\) se:
\begin{itemize}
\item \(G\) è \textsc{derivabile}
\item \(\forall x \in [a, b] G' = f(x)\)
\end{itemize}

\subsection{Proposizione}
Due primitive della stessa funzione definite sullo stesso intervallo differiscono per una costante.

\paragraph{Dimostrazione}
\(G_1, G_2\) primitive di \(f\)
\[\forall x \in \mathbb{R}, G_1'(x) = f(x), G_2'(x) = f(x)\]
\[G_1'(x) - G_2'(x) = 0\]
\[(G_1 - G_2)'(x) = 0\]
\[G_1 = G_2 + C\]

\subsubsection{Se non si è su un intervallo...}
Esistono primitive di una funzione che non differiscono per una costante, ma per qualcosa di più.
\paragraph{Esempio}
\[G_1(x) = \begin{cases}
    log(x) \qquad se\ x > 0\\
    log(-x) \qquad se\ x < 0
\end{cases}\]

\[G_2(x) = \begin{cases}
    1 + log(x) \qquad se\ x > 0\\
    log(-x) \qquad se\ x < 0
\end{cases}\]

\subsection{Funzioni senza primitiva}
\[\delta(x)\qquad delta\ di\ Dirac\]
\paragraph{Dimostrazione}
Per assurdo, immaginiamo esista una primitiva \(F\) di \(f\).\\
Negli intervalli \(]-\infty, 0[\) e \(]0, +\infty[\) si ha che \(F'(x) = 0\), e quindi che la funzione è costante.\\
Se la funzione è una \textsc(primitiva), significa che dev'essere \textsc{derivabile}, e quindi \textsc{continua}.\\
Ma la funzione originale non è continua, perchè ha un salto in \(x = 0\). Assurdo.

\section{Integrale indefinito}
\[\int f(x) dx\]
L'integrale indefinito qui sopra indica l'insieme di tutte le primitive di \(f(x)\).\\
\\
Esistono funzioni che hanno primitiva, ma non è esprimibile:
\[\int \frac{sin t}{t} dt\]

\end{document}