\documentclass{article}
\usepackage[utf8]{inputenc}
\usepackage{mathtools}
\usepackage{amssymb}
\usepackage{centernot}
\usepackage{bm}
\usepackage{fullpage}

% Iniziate a scrivere da qua in poi
\begin{document}

\section{Moltiplicazioni tra matrici}

\[
    \begin{bmatrix}
        a & b \\
        c & d \\
    \end{bmatrix}
    * 
    \begin{bmatrix}
        e & f \\
        g & h \\
    \end{bmatrix}
    =
    \begin{bmatrix}
        ae + cf & be + df \\
        ag + ch & bg + dh \\
    \end{bmatrix}
\]

\section{Invertibilità di una matrice}

Si può verificare se una matrice \( A \) quadrata di ordine \( n \) è invertibile verificando una di queste definizioni equivalenti:

\begin{itemize}
\item Il determinante non è nullo: \( \det A\neq 0 \).
\item Il rango di \( A \) è \( n \).
\item La trasposta \( A^{T} \) è una matrice invertibile.
\item Tutte le righe/colonne di \( A \) sono linearmente indipendenti.
\item Tutte le righe/colonne di \( A \) formano una base di \( \mathbb{K} ^{n} \).
\item Il numero 0 non è un autovalore di \( A \).
\item \( A \) è trasformabile mediante algoritmo di Gauss-Jordan in una matrice con \( n \) pivot.
\end{itemize}

\section{Stabilire esistenza di funzione lineare}

Per controllare se esiste o no una funzione lineare è sufficiente verificare che sia valida la proprietà di linearità:\\
\begin{itemize}
\item Se due vettori sono linearmente indipendenti, anche i risultati della funzione devono essere linearmente indipendenti.
\end{itemize}
Può essere controllata velocemente vedendo se si verificano le seguenti condizioni:
\begin{itemize}
\item Se due vettori di ingresso sono uno multiplo dell'altro, allora anche i vettori di uscita devono essere uno multiplo dell'altro per la stessa costante.
\item Se un vettore di ingresso è dato dalla somma di (multipli di) altri, allora anche il vettore di uscita deve essere dato dalla somma di (multipli degli) stessi.
\end{itemize}

\section{Determinazione di matrice associata}

Vogliamo trovare la matrice associata (\(A\)) di una funzione rispetto a delle nuove basi, ad esempio \(< (1, 2, 3), (4, 5, 6), (7, 8, 9)\).\\

Procediamo disponendo in verticale gli elementi delle basi, in questo modo:
\[
    M = 
    \begin{matrix}
        1 & 4 & 7 \\
        2 & 5 & 8 \\
        3 & 6 & 9 \\
    \end{matrix}
\]

Troviamo la matrice inversa con il metodo di Gauss-Jordan:
\[
...
\]

Calcoliamo il risultato di:
\[
    B = M^{-1} * A * M
\]

Il risultato \(B\) sarà la nostra nuova matrice associata.

\section{Diagonalizzabilità}
Una matrice è \textsc{diagonalizzabile} se ha \textbf{tanti autovalori quanto il suo rango}.\\
Per trovare gli autovalori trovare dove il polinomio caratteristico (determinante della matrice fatta come quella qui sotto) è uguale a 0:
\[
    \begin{vmatrix}
        1 - x & 2     & 3 \\
        4     & 5 - x & 6 \\
        7     & 8     & 9 - x \\
    \end{vmatrix}
    = 0
\]

\section{Stabilire se una funzione è lineare}

Se tutti i termini della funzione sono \textbf{polinomi omogenei} di primo grado (non ci sono potenze superiori a 1), allora è automaticamente \textsc{lineare}.

\section{Immagine}

Le \textsc{basi dell'immagine} di una funzione sono i \textbf{vettori linearmente indipendenti} che la generano.

\section{Iniettività e suriettività}

Una funzione lineare è \textsc{iniettiva} se \textbf{il nucleo è di dimensione 0}, ovvero se l'unico valore che fa risultare 0 alla funzione è il vettore nullo.\\
\\
Una funzione lineare è \textsc{suriettiva} se la dimensione dell'immagine è minore o uguale al rango della funzione (degli input, il rango della matrice associata): \(dim(Im(F)) = rk(M_F)\).\\

\subsection{Matrici quadrate}

Se la funzione è un \textbf{automorfismo} (campo input = campo output), allora \(iniettivita' \Leftrightarrow suriettivita'\).

\section{Somma diretta}

Un sottospazio è \textsc{somma diretta} se i due sottospazi di cui viene fatta la somma \textbf{non hanno basi in comune}, e quindi \(dim(\pmb{U} \cap \pmb{W}) = 0\).

\subsection{Trovare basi che diano una somma diretta}

Per trovare basi che diano una somma diretta, è sufficiente \textbf{trovare basi linearmente indipendenti} con quelle che già abbiamo: solitamente parti della base canonica funzionano alla perfezione.

\end{document}