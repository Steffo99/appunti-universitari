\documentclass{article}
\usepackage[utf8]{inputenc}
\usepackage{mathtools}
\usepackage{amssymb}
% New symbols
\let\oldsqrt\sqrt
\def\sqrt{\mathpalette\DHLhksqrt}
\def\DHLhksqrt#1#2{%
\setbox0=\hbox{$#1\oldsqrt{#2\,}$}\dimen0=\ht0
\advance\dimen0-0.2\ht0
\setbox2=\hbox{\vrule height\ht0 depth -\dimen0}
{\box0\lower0.4pt\box2}}
% End new symbols

\begin{document}

\section{Vettori}
Un vettore è una struttura costituita da \textbf{n scalari}, tutti nello \textbf{stesso campo numerico} \(\mathbb{K}\).\\
Possiamo chiamare un vettore costituito da n scalari una \textbf{n-upla} ("ennupla").\\
Ad esempio, diciamo che un vettore costituito da 3 numeri naturali è in \(\mathbb{N}^3\), e lo rappresentiamo scrivendo \(\mathbf{v} = (3, 5, 12)\).\\

\section{Spazi vettoriali}
Uno spazio vettoriale è una struttura costituita da \textbf{un campo numerico}, \textbf{un insieme di vettori} non vuoto e le operazioni di \textbf{somma} e \textbf{prodotto per scalare}.\\
\\
Si dice che un vettore \textbf{appartiene} allo spazio vettoriale se questo è presente all'interno dell'insieme dello spazio vettoriale.\\
Tutti i vettori appartenenti allo spazio sono tutti definiti nello \textbf{stesso campo numerico}: non è possibile che un vettore appartenga ad uno spazio definito nel campo \(\mathbb{K}\) e sia esso stesso definito nel campo \(\mathbb{L}\).\\
\\
La somma in uno spazio vettoriale \(\mathbf{v} + \mathbf{w}\) è tra due vettori appartenenti a quest'ultimo; il prodotto per scalare \(\alpha \mathbf{v}\) invece è tra un vettore appartenente allo spazio vettoriale e uno degli scalari del campo dello spazio vettoriale.
Le proprietà della somma e del prodotto sono le stesse che siamo abituati a vedere normalmente.\\\\
Per l'addizione:
\begin{itemize}
    \item Commutativa \(\mathbf{a} + \mathbf{b} = \mathbf{b} + \mathbf{a}\)
    \item Associativa (e dissociativa) \((\mathbf{a} + \mathbf{b}) + \mathbf{c} = \mathbf{a} + (\mathbf{b} + \mathbf{c})\)
    \item Esistenza dell'opposto \(\mathbf{a} + (-\mathbf{a}) = \mathbf{0}\)
    \item Esistenza del neutro \(\mathbf{a} + \mathbf{0} = \mathbf{a}\)
\end{itemize}
Per la moltiplicazione tra vettore e scalare:
\begin{itemize}
    \item Associativa \((\alpha \beta) \mathbf{a} = \alpha (\beta \mathbf{a})\)
    \item Esistenza dello scalare nullo \(0 \mathbf{a} = \mathbf{0}\)
    \item Esistenza dello scalare neutro \(1 \mathbf{a} = \mathbf{a}\)
    \item Distributività per vettori \(\alpha (\mathbf{a} + \mathbf{b}) = \alpha \mathbf{a} + \alpha \mathbf{b}\)
    \item Distributività per scalari \((\alpha + \beta) \mathbf{a} = \alpha \mathbf{a} + \beta \mathbf{a}\)
\end{itemize}

\section{Sottospazi vettoriali}
Un sottospazio vettoriale è una struttura che rappresenta \textbf{un sottoinsieme di spazio vettoriale}.\\
Perchè uno spazio vettoriale \(\mathbf{W}\) sia effettivamente sottospazio di un altro spazio \(\mathbf{V}\), deve soddisfare i seguenti requisiti:
\begin{itemize}
    \item I due spazi sono definiti nello stesso campo
    \item Tutti i vettori appartenenti a \(\mathbf{W}\) sono presenti anche in \(\mathbf{V}\)
    \item \(\mathbf{W}\) contiene tutti i possibili vettori risultanti da somma e prodotto (e quindi da combinazioni lineari) dei suoi elementi
\end{itemize}

\section{Sistema di generatori}
Un sistema di generatori per uno spazio vettoriale è un insieme di vettori che tramite una loro combinazione lineare possono dare come risultato un qualsiasi elemento di uno spazio.
[TODO]

\section{Base di spazio vettoriale}
Una base di uno spazio vettoriale è [TODO]

\end{document}
